\def\mySecNum{20.2}
\mySection{\mySecNum~Brownian motion}
%-------------- start slide -------------------------------%{{{ 1
\begin{frame}[fragile,t]
\begin{mydefinition}
	A real-valued stochastic process $Z(t)$ is called a \textcolor{magenta}{Brownian motion} or \textcolor{magenta}{Wiener process} if
	\begin{enumerate}
		\item It starts at $0$:
			 \begin{align*}
				 Z(0)=0.
			\end{align*}
		\item  For $0\le s <t$, the increment $Z(t)-Z(s)$ is normally distributed with mean zero and
			variance $t-s$:
			\begin{align*}
				Z(t)-Z(s)\sim N(0, t-s).
			\end{align*}
		\item Its increments are independent: if
			\begin{align*}
				0\le t_0 \le t_1 \le \cdots \le t_k,
			\end{align*}
			then
			\begin{align*}
				\bbP\left(Z(t_i)-Z(t_{i-1})\in H_i, \: 1\le i\le k\right)
				= \prod_{i=1}^k \bbP\left(Z(t_i)-Z(t_{i-1})\in H_i\right).
			\end{align*}
	\end{enumerate}
\end{mydefinition}
\vfill
\pause
\begin{remark}
	One can always construct a \textcolor{cyan}{continuous version} of the Brownian motion; from now
	on, we always assume that Brownian motion is a continuous process.
\end{remark}
\end{frame}
%-------------- end slide -------------------------------%}}}
%-------------- start slide -------------------------------%{{{ 1
\begin{frame}[fragile,t]
	\begin{mythm}[(Some properties of Brownian motion)]
		\begin{enumerate}
			\item $Z(t)$ is \textcolor{cyan}{nowhere differentiable}.
			\item[] \textcolor{gray}{(Hence, $dZ(t)$ requires some special treatment.)}
				\bigskip
			\item $Z(t)$ satisfies the \textcolor{cyan}{scaling property}:
			\item[]
				\begin{center}
					$\widetilde{Z}(t):=\frac{1}{\sqrt{c}} Z(ct)$ is also a B.M. for all $c>0$.
				\end{center}

			\item $Z(t)$ is a \textcolor{cyan}{martingale}, namely,
				\begin{align*}
					\E\left(Z(t+s)|Z(t)\right) = Z(t).
				\end{align*}

			\item For any $t>0$,  $Z(t)\sim N(0,t)$ and
				\begin{align*}
					\E(Z(t)Z(s)) = \min(t,s)	\quad \text{for all $t,s\ge 0$}.
				\end{align*}

			\item $Z(t)$ is \textcolor{cyan}{translation invariant}, namely,
				\begin{center}
					$\widetilde{Z}(t):=Z(t+t_0)-Z(t_0)$ is also a B.M. for all $t_0\ge 0$.
				\end{center}
		\end{enumerate}
	\end{mythm}
\end{frame}
%-------------- end slide -------------------------------%}}}
%-------------- start slide -------------------------------%{{{ 1
\begin{frame}[fragile,t]
\begin{myproof}
	Part (1) goes beyond this course. All the rest could be proved using our current knowledge.
	\vfill

	\myEnd
\end{myproof}
\end{frame}
%-------------- end slide -------------------------------%}}}
%-------------- start slide -------------------------------%{{{ 1
\begin{frame}[fragile,t]
	\frametitle{Arithmetic Brownian motion}
\begin{mydefinition}
	Let $Z(t)$ be a B.M. Then the process $X(t)$ given by
	 \begin{align*}
		 dX(t) = \alpha dt+ \sigma d Z(t)
	\end{align*}
	is called an \textcolor{magenta}{arithmetic Brownian motion}. Equivalently, $X(t)$ can be written
	in the following integral representation:
\begin{align*}
	X(t) = X(0) + \int_0^t \alpha ds + \int_0^t \sigma dZ(s).
\end{align*}
\end{mydefinition}
\end{frame}
%-------------- end slide -------------------------------%}}}
%-------------- start slide -------------------------------%{{{ 1
\begin{frame}[fragile,t]
	\begin{remark}
		\begin{enumerate}
			\item $X(t)$ is normally distributed:
				 \begin{align*}
					 X(t) = \sigma t + \sigma Z(t) \sim N\left(\sigma t, \sigma^2 t\right).
				\end{align*}

			\item $X(t)$ takes both positive and negative values almost surely.

			\item $\alpha t$ is a drift term.
		\end{enumerate}
	\end{remark}
\end{frame}
%-------------- end slide -------------------------------%}}}
%-------------- start slide -------------------------------%{{{ 1
\begin{frame}[fragile,t]
	\frametitle{The Ornstein-Uhlenbeck process}
\begin{mydefinition}
	Let $Z(t)$ be a B.M. Then the process $X(t)$ given by
	 \begin{align*}
		 dX(t) = \lambda \left(\alpha - X(t)\right) dt+ \sigma d Z(t)
	\end{align*}
	is called the \textcolor{magenta}{Ornstein-Uhlenbeck process}.
\end{mydefinition}
\end{frame}
%-------------- end slide -------------------------------%}}}
%-------------- start slide -------------------------------%{{{ 1
\begin{frame}[fragile,t]
\begin{remark}
	Equivalently, $X(t)$ can be written in the following integral representation:
	\begin{align*}
		X(t) = X(0) + \lambda \int_0^t \left(\alpha - X(s)\right) ds + \int_0^t \sigma dZ(s),
	\end{align*}
	which is an integral equation (unknown $X$ appears on both sides).
\end{remark}
\bigskip
\begin{remark}
	We have introduced \textcolor{magenta}{mean reversion} in the drift term.
\end{remark}
\end{frame}
%-------------- end slide -------------------------------%}}}
