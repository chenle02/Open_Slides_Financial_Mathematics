\def\mySecNum{13.5}
\mySection{\mySecNum~The Black-Scholes analysis}
%-------------- start slide -------------------------------%{{{ 1 BS equation
\begin{frame}[fragile,t]
 From the previous section we see that
  \begin{gather*}
    \text{Market-maker's profit}
    =-\left(\frac{1}{2}\sigma^2 S_t^2 \Gamma_t + \theta_t + r \left[ \Delta_t S_t -C(S_t)\right]\right)h
   \end{gather*}
   \bigskip

   If one believes that via one-standard deviation move, the market-maker's profit is approximately zero, we arrive at the \textcolor{magenta}{Black-Scholes equation:}

   \bigskip
   \begin{align*}
      \boxed{\frac{1}{2}\sigma^2 S_t^2 \Gamma_t + \theta_t + r \Delta_t S_t  = r C(S_t)}
   \end{align*}
\end{frame}
%-------------- end slide -------------------------------%}}}
%-------------- start slide -------------------------------%{{{ 1 PDF form
\begin{frame}[fragile,t]
  Equivalently, this can be written as a standard PDE:
  \begin{align*}
    \mathcal{L}_{\text{BS}} V(t,S) = 0
  \end{align*}
  where $V(t,S)$ refers to option (call or put) price and
  \begin{align*}
    \mathcal{L}_{\text{BS}} =
    \frac{\partial }{\partial t} + \frac{1}{2} \sigma^2 S^2 \frac{\partial^2 }{\partial S^2}
    + r\: S \frac{\partial}{\partial S} V(t,S) -r.
  \end{align*}

  \bigskip

  One still needs to put the correct boundary conditions.
\end{frame}
%-------------- end slide -------------------------------%}}}
%-------------- start slide -------------------------------%{{{ 1 Assumptions
\begin{frame}[fragile,t]
  \begin{itemize}
    \item Under the following assumptions:
    \item[] Underlying asset and the option do not pay dividends
    \item[] Interest rate and volatility are constant
    \item[] The stock does not make large discrete moves
      \bigskip
    \item The equation is valid only when early exercise is not optimal
  \end{itemize}
\end{frame}
%-------------- end slide -------------------------------%}}}
