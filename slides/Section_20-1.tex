\def\mySecNum{20.1}
\mySection{\mySecNum~The Black-Scholes assumption about stock prices}
%-------------- start slide -------------------------------%{{{ 1
\begin{frame}[fragile,t]
	The vast majority of technical option pricing discussions, including the original paper by Black
	and Scholes, assume that the price of the underlying asset follows a process determined by
	\bigskip

	\begin{align}
		\label{E:SDE}
		dS(t) = (\alpha-\delta) dt+ \sigma d Z(t), \quad  S(0)=S_0.
	\end{align}
\end{frame}
%-------------- end slide -------------------------------%}}}
%-------------- start slide -------------------------------%{{{ 1
\begin{frame}[fragile,t]
\begin{align*}
	\boxed{dS(t) = (\alpha-\delta) dt+ \sigma d Z(t), \quad  S(0)=S_0}
\end{align*}
\bigskip
\begin{itemize}
	\item $S(t)$ is the \textcolor{cyan}{stock price}. $d S(t)$	is the instantaneous change in the
		stock price. $S_0$ is the initial asset value.
	\item $\alpha$ is the \textcolor{cyan}{continuously compound expected return} on the stock;
	\item $\sigma$ is the \textcolor{cyan}{volatility}, i.e., the standard deviation of the instantaneous return;
	\item $Z(t)$ is the \textcolor{cyan}{standard Brownian motion}.
	\item $dZ(t)$ requires rigorous justification.
\end{itemize}
\bigskip
\mySeparateLine
\bigskip
\begin{itemize}
	\item Equation of this type is called \textcolor{magenta}{stochastic differential equation}.
	\item Solution to this specific equation is the \textcolor{magenta}{geometric Brownian motion}.
\end{itemize}
\end{frame}
%-------------- end slide -------------------------------%}}}
%-------------- start slide -------------------------------%{{{ 1
\begin{frame}[fragile,t]
	\begin{remark}
		\label{R:lognormal}
		We will see in this chapter that solution to this equation is lognormally distributed:
		\begin{align*}
			\ln(S(t)) \sim N\left(\ln(S_0)+ \left(\alpha-\delta-\frac{1}{2}\sigma^2\right)t, \: \sigma^2\:
			t\right), \quad \text{for all $t>0$}.
		\end{align*}
	\end{remark}
	\bigskip
	\bigskip
	\pause
	\begin{remark}
		 Note that Remark \myref{R:lognormal} is valid for all $t>0$. It works for the terminal time
		 $t=T$. It can also help us solve path-dependent options.
	\end{remark}
\end{frame}
%-------------- end slide -------------------------------%}}}
