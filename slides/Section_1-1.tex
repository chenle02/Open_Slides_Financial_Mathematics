\def\mySecNum{1.1}
\mySection{\mySecNum~What is a derivative?}
%-------------- start slide -------------------------------%{{{ 1
\begin{frame}[fragile,t]
	\begin{mydefinition}
		A \textcolor{magenta}{derivative} is a financial instrument that has a value determined by the price of something
		else.
	\end{mydefinition}
\end{frame}
%-------------- end slide -------------------------------%}}}
%-------------- start slide -------------------------------%{{{ 1
\begin{frame}[fragile,t]
	\begin{myexample}
		An agreement where
		\begin{center}
			you pay \$1 if the price of corn is greater than \$3 \\
			and                                                  \\
			you receive \$1 if the price of corn is less that \$1
		\end{center}
		is a derivative.
	\end{myexample}

	\bigskip
	\mySeparateLine
	\bigskip

	\begin{center}
		This contract can be used to   \\
		\bigskip
		\textcolor{magenta}{speculate} on the price of corn \\
		or                             \\
		it can be used to \textcolor{cyan}{reduce risk}.
	\end{center}
	\bigskip

	Hence, it is not the contract itself, but how it is used, and who uses it, that determines whether
	or not it is risk-reducing. It all depends on context.
\end{frame}
%-------------- end slide -------------------------------%}}}
